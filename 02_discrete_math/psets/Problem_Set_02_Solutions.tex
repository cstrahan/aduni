\documentclass[12pt]{amsart}

\setlength{\parsep}{3pc}
\setlength{\itemsep}{0.2in}


\usepackage{fullpage}
\usepackage{psfig}

\title[Problem Set 2]{Ars Digita University\\Month 2:  Discrete Mathematics\\Professor Shai Simonson\\Problem Set 2 SOLUTIONS --- Set, Functions, Big-O, Rates of Growth}

\begin{document}

\maketitle

\begin{enumerate}
\item {\bf Prove by formal logic:}
\begin{enumerate}
\item {\bf The complement of the union of two sets equals the intersection of the complements.}

Let $x \in \overline{(A \cup B)}$.  Then $x$ is not in $A$, and $x$ is not in $B$.  Therefore $x \in \overline{A} \cap \overline{B}$.  Since $x$ was an arbitrary element of $\overline{(A \cup B)}$, we conclude that $\overline{(A \cup B)} \subseteq \overline{A} \cap \overline{B}$.

Now, suppose that $x \in \overline{A} \cap \overline{B}$.  We can conclude that $x \in \overline{A}$, and that $x \in \overline{B}$.  This impiles that $x \notin A$, and $x \notin B$, which implies $x \notin (A \cup B)$, which implies that $x \in \overline{(A \cup B)}$.  Again, since $x$ was arbitrary, we conclude that $\overline{(A \cup B)} \subseteq \overline{A} \cap \overline{B}$.  Two sets which are subsets of each other are equal (see below), so we can conclude that $\overline{(A \cup B)} = \overline{A} \cap \overline{B}$

\item {\bf The complement of the interesection of two sets equals the union of the complements.}

Let $x \in \overline{(A \cap B)}$.  Then $x$ cannot be in both $A$ and
$B$ --- at least one of the statements ``$x \in \overline{A}$'', ``$x
\in \overline{B}$'' is true.  Therefore $x \in \overline{A} \cup \overline{B}$, and $\overline{(A \cap B)} \subseteq \overline{A} \cup \overline{B}$.

Now let $x \in \overline{A} \cup \overline{B}$.  Again, $x$ cannot be
in both $A$ and $B$, so $x \notin A \cap B$, so $x \in \overline(A
\cap B)$, and $\overline{A} \cup \overline{B} \subseteq \overline{(A
\cap B)}$.  The two sets are equal.

\item {\bf $(B-A) \cup (C-A) = (B \cup C) - A$.}

Let $x \in (B-A) \cup (C-A)$.  Then $x \in (B-A) \lor x \in (C-A)$.
Assume without loss of generality that $x \in (B-A)$.  This implies
that $x \in B$ and $x \notin A$.  Which implies that $x \in (B \cup
C)$, and, combining this with the fact that $x \notin A$, we conclude
that $x \in (B \cup C) - A$, and $(B-A) \cup (C-A) \subseteq (B \cup
C) - A$.  If $x \in (B \cup C) - A$, then $x \notin A$, and $x \in B
\lor x \in C$.  Therefore $x \in (B-A) \lor x \in (C-A)$, $x \in (B-A)
\cup (C-A)$, and $(B \cup C) - A \subseteq (B-A) \cup (C-A)$,
completing the proof.

\medskip

\item {\bf If two sets are subsets of each other than they are equal.}

If two sets $A$ and $B$ are not equal, then there exists some element
that is in one set that is not in the other.  Without loss of
generality, denote such an element $x$ and assume that $x \in A$.
Since $x \in A$ and $x \notin B$, we conclude that $A \not\subset B$.

\end{enumerate}

\medskip

\item {\bf Generalize De Morgan's laws for $n$ sets and prove the laws by induction.}

The $n$-version generalization of De Morgan's laws are as follows:
\begin{eqnarray*}
\overline{\bigcup_{i=1}^n A_i} & = & \bigcap_{i=1}^n \overline{A_i} \\
\overline{\bigcap_{i=1}^n A_i} & = & \bigcup_{i=1}^n \overline{A_i} \\
\end{eqnarray*}

To prove these laws by induction, we use $n=2$ as our base cases:
these are the traditional two-set De Morgan's laws.  Now, we assume
that the laws hold for $n$ sets, and show that they hold for $n+1$
sets as well, using both our inductive hypothesis and the two-set De
Morgan's laws:

\begin{eqnarray*}
\overline{\bigcup_{i=1}^{n+1} A_i} & = & \overline{(\bigcup_{i=1}^n A_i) \cup A_{n+1}} \\
& = & \overline{(\bigcup_{i=1}^n A_i)} \cap \overline{A_{n+1}} \\
& = & (\bigcap_{i=1}^n \overline{A_i}) \cap \overline{A_{n+1}} \\
& = & \bigcap_{i=1}^{n+1} \overline{A_i} \\
& & \\
\overline{\bigcap_{i=1}^{n+1} A_i} & = & \overline{(\bigcap_{i=1}^n A_i) \cap A_{n+1}} \\
& = & \overline{(\bigcap_{i=1}^n A_i)} \cup \overline{A_{n+1}} \\
& = & (\bigcup_{i=1}^n \overline{A_i}) \cup \overline{A_{n+1}} \\
& = & \bigcup_{i=1}^{n+1} \overline{A_i} \\
\end{eqnarray*}

\medskip

\item {\bf Prove by induction on the size of the set, that the power set $P(A)$ has cardinality $2^{|A|}$.}

For a set S containing a single element $x$, the power set is
$\{\emptyset, \{x\}\}$, which is of size 2.  This is our base case.
Now assume that a set $A$ of size $n$ has power set with cardinality
$2^{|A|} = 2^n$.  Consider adding a new element $a$ to $A$ to make the
set $A'$, of size $n+1$.  The power set of $A'$ will consist of all
the sets in the power set of $A$, plus all those sets taken again,
with the element $a$ added.  Every set in $P(A)$ gives rise to two
sets in $P(A')$.  We conclude that $|P(A')| = |2P(A)| = 2(2^n) =
2^{n+1} = 2^{|A'|}$.

\medskip

\item {\bf $A \oplus B$ is defined to be the set of all elements in $A$ or $B$ but not in both $A$ and $B$.}
\begin{enumerate}
\item {\bf Determine whether or not $\oplus$ is commutative.  Prove your answer.}

We check this by means of a membership table.

\begin{tabular}{|c|c|c|c|}
\hline
$A$ & $B$ & $A \oplus B$ & $B \oplus A$ \\
\hline
1 & 1 & 0 & 0 \\
1 & 0 & 1 & 1 \\
0 & 1 & 1 & 1 \\
0 & 0 & 0 & 0 \\
\hline
\end{tabular}

We conclude that $\oplus$ is commutative.

\item {\bf Determine whether $\oplus$ is associative.  Prove your answer.}

\begin{tabular}{|c|c|c|c|c|c|c|}
\hline
$A$ & $B$ & $C$ & $A \oplus B$ & $B \oplus C$ & $(A \oplus B) \oplus C$ & $A \oplus (B \oplus C) $ \\
\hline
1 & 1 & 1 & 0 & 0 & 1 & 1 \\
1 & 1 & 0 & 0 & 1 & 0 & 0 \\
1 & 0 & 1 & 1 & 1 & 0 & 0 \\
1 & 0 & 0 & 1 & 0 & 1 & 1 \\
0 & 1 & 1 & 1 & 0 & 0 & 0 \\
0 & 1 & 0 & 1 & 1 & 1 & 1 \\
0 & 0 & 1 & 0 & 1 & 1 & 1 \\
0 & 0 & 0 & 0 & 0 & 0 & 0 \\
\hline
\end{tabular}

We conclude that $\oplus$ is associative; indeed, $\oplus$ acts as a
parity operator: if we ``$\oplus$'' $n$ sets together, we will get a
set consisting of all those elements that are in an odd number of our
sets.


\item {\bf Determine whether $\oplus$ can be distributed over union.  Prove your answer.}

\begin{tabular}{|c|c|c|c|c|c|c|c|}
\hline
$A$ & $B$ & $C$ & $B \cup C$ & $A \oplus (B \cup C)$ & $A \oplus B$ & $A \oplus C$ & $(A \oplus B) \cup (A \oplus C)$ \\
\hline
1 & 1 & 1 & 1 & 0 & 0 & 0 & 0 \\
1 & 1 & 0 & 1 & 0 & 0 & 1 & 1 \\
1 & 0 & 1 & 1 & 0 & 1 & 0 & 1 \\
1 & 0 & 0 & 0 & 1 & 1 & 1 & 1 \\
0 & 1 & 1 & 1 & 1 & 1 & 1 & 1 \\
0 & 1 & 0 & 1 & 1 & 1 & 0 & 1 \\
0 & 0 & 1 & 1 & 1 & 0 & 1 & 1 \\
0 & 0 & 0 & 0 & 0 & 0 & 0 & 0 \\
\hline
\end{tabular}

Looking at the second and third lines, we see that $\oplus$ does not
distribute over union.  In particular, if an element $x$ is in a set
$A$ and is in {\em one} of $B$ or $C$, {\em but not both}, then $x \in
(A \oplus B) \cup (A \oplus C)$, but $x \notin A \oplus (B \cup C)$.

\item {\bf Determine whether $\oplus$ can be distributed over intersection.  Prove your answer.}

\begin{tabular}{|c|c|c|c|c|c|c|c|}
\hline
$A$ & $B$ & $C$ & $B \cap C$ & $A \oplus (B \cap C)$ & $A \oplus B$ & $A \oplus C$ & $(A \oplus B) \cap (A \oplus C)$ \\
\hline
1 & 1 & 1 & 1 & 0 & 0 & 0 & 0 \\
1 & 1 & 0 & 0 & 1 & 0 & 1 & 0 \\
1 & 0 & 1 & 0 & 1 & 1 & 0 & 0 \\
1 & 0 & 0 & 0 & 1 & 1 & 1 & 1 \\
0 & 1 & 1 & 1 & 1 & 1 & 1 & 1 \\
0 & 1 & 0 & 0 & 0 & 1 & 0 & 0 \\
0 & 0 & 1 & 0 & 0 & 0 & 1 & 0 \\
0 & 0 & 0 & 0 & 0 & 0 & 0 & 0 \\
\hline
\end{tabular}

Reasoning as above, we see that $\oplus$ does not distribute over
intersection.  If an element $x$ is in a set $A$ and is in {\em one}
of $B$ or $C$, {\em but not both}, then $x \in A \oplus (B \cap C)$,
but $x \notin (A \oplus B) \cap (A \oplus C)$.

\end{enumerate}

\medskip

\item {\bf Assume a universal set of 8 elements.  Given a set $A = a_7a_6a_5a_4a_3a_2a_1a_0$, represented by 8 bits, explain how to use bitwise and/or/not operations, in order to:}
\begin{enumerate}
\item {\bf Extract the rightmost bit of set $A$.}

To ``extract'' the rightmost bit of set $A$, we compute $A \land 00000001$.

\item {\bf Extract the odd numbered bits.}
To ``extract'' the odd numbered bits, we compute $A \land 10101010$.

\item {\bf Make bits 4-6 equal to 1.}
To set bits 4-6 equal to 1, without changing anything else, we compute $A \lor 01110000$.
\end{enumerate}
{\bf Given another set $B$, explain how to:}
\begin{enumerate}
\item {\bf Determine if $A \subseteq B$.}
$A \subseteq B$ iff $A - B = 0$ (and see below)

\item {\bf Extract $A - B$.}
$A - B$ can be computed as $A \land (\overline{A \land B}$).
\end{enumerate}

\medskip

\item {\bf The Inclusion/Exclusion theorem for three sets says that if there are three sets $A$, $B$, and $C$, then

$$
|A \cup B \cup C| = |A| + |B| + |C| - |A \cap C| - |A \cap C| - |B \cap C| + |A \cap B \cap C|
$$

Note that the converse of this theorem is not true.

Given a set of 29 students, where 8 need housing and financial aid; 12
need housing, financial aid, and an e-mail account; 17 need an e-mail
account and financial aid; 23 need housing, 20 need financial aid, 19
need e-mail accounts and 4 students don't need anything:
}

\begin{enumerate}
\item {\bf How many students need both housing and e-mail?}

Let $H$, $F$, and $E$ denote the sets of students needing housing,
financial aid, and an e-mail account, respectively.  We are trying to
compute $|H \cap E|$ from the Inclusion/Exclusion principal, given
everything else:

\begin{eqnarray*}
|H \cap E| & = & |H| + |F| + |E| - |H \cap F| - |F \cap E| + |H \cap F \cap E| - |H \cup F \cup E| \\
& = & 23 + 20 + 19 - 8 - 17 + 12 - 25 \\
& = & 24 \\
\end{eqnarray*}

\item {\bf Change the numbers to 57,8,3,21,21,32,31,8.  What is your answer now?}
\begin{eqnarray*}
|H \cap E| & = & 21 + 32 + 31 - 8 - 21 + 3 - 49 \\
& = & 3 \\
\end{eqnarray*}
\item {\bf Which answer is bogus and why?}

The first answer is bogus.  It is impossible to have 24 students
requiring housing and an email account, when only 23 students require
housing.  Sets that satisfy the characteristics given in the first
part of the problem do not exist.

\end{enumerate}

\item {\bf How many numbers are there between 1 and 10,000, which are either even, end in 0, or have the sum of their digits divisible by 9?  Hint: Use inclusion/exclusion.}

Let $E$, $Z$ and $N$ denote the numbers between 1 and 10,000 which are
even, end in 0, or have the sum of their digits divisible by 9,
respectively (not that $N$ consists of all {\em multiples of 9}.
Then, (including the number 10,000 in our calculations), $|E| = 5000,
|Z| = 1000, |N| = 1111, |E \cap Z| = 1000, |E \cap N| = 555, |Z \cap
N| = 111, |E \cap Z \cap N| = 111$.  Therefore, by the
Inclusion/Exclusion principle:

\begin{eqnarray*}
|E \cup Z \cup N| & = & 5000 + 1000 + 1111 - 1000 - 555 - 111 + 111 \\
1& = & 5556
\end{eqnarray*}

\medskip

\item {\bf Prove that $f(x) = x^3 - 1000x^2 + x - 1$ is $\Omega(x^3)$ and $O(x^3)$.}

For all $x > 10,000$:

\begin{eqnarray*}
f(x) & = & x^3 - 1000x^2 + x - 1 \\
& > & x^3 - 1000x^2 \\
& = & (x-1000)x^2 \\
& > & (.9x)x^2 \\
& = & .9x^3 \\
\end{eqnarray*}

Therefore, $f(x)$ is $\Omega(x^3)$ with $C = .9, k = 10,000$.

Also, for all $x > 0$:

\begin{eqnarray*}
f(x) & = & x^3 - 1000x^2 + x - 1 \\
& < & x^3 + 1000x^3 + x^3 + x^3 \\
& = & 1002x^3 \\
\end{eqnarray*}

Therefore, $f(x)$ is $O(x^3)$ with $C = 1002, k = 1$.

Note that these result are a direct consequence of Theorem 4 on page
90 of Rosen.

\medskip

\item {\bf Prove that $1^k + 2^k + 3^k + \ldots + n^k$ is $\Omega(n^{k+1})$ and $O(n^{k+1}$, for any constant $k$.}

\begin{eqnarray*}
1^k + 2^k + 3^k + \ldots + n^k & > & (\frac{n}{2})^k + (\frac{n}{2}+1)^k + \ldots + n^k \\
& > & \underbrace{(\frac{n}{2})^k + (\frac{n}{2})^k + \ldots + (\frac{n}{2})^k}_{\frac{n}{2} \ \tt{terms}} \\
& = & \frac{n}{2}(\frac{n}{2})^k \\
& = & (\frac{n}{2})^{k+1} \\
\end{eqnarray*}

Therefore, $f(n) = 1^k + 2^k + 3^k + \ldots + n^k$ is
$\Omega(n^{k+1}): f(n) > C(n^{k+1})$, where $C = \frac{1}{2^k}$.  Note
that this constant depends on $k$.  $f(n)$ is also $O(n^{k+1})$:

\begin{eqnarray*}
1^k + 2^k + 3^k + \ldots + n^k & < & \underbrace{n^k + n^k + n^k + \ldots + n^k}_{n \ \tt{terms}} \\
& = & n(n^k) \\
& = & n^{k+1} \\
\end{eqnarray*}

{\bf \item Order the following functions in order of their growth rate.  If $f(x)$ is $O(g(x))$, but $g(x)$ is not $O(f(x))$, then put $f(x)$ above $g(x)$.  If they are each big-$O$ of each other, then place them on the same level.

\medskip
\begin{math}
\begin{array}{llllllll}
x^2 + x^3 & 3^x & x! & \frac{x}{log_2 x} & x^2+2^x & x \log_2 x & 2^{x\log x} & \log x^2 \\
\log_2 x! & log_2 x & \ln x & 2^x & x(1+2+\ldots+x) & \log \log x & 2^{x^2} & \log^2 x \\
\end{array}
\end{math}
}

We list functions in increasing order of growth.  Functions on the
same line have identical orders of growth.

\begin{center}
$2^{x^2}$ \\
$x!$ \ \ \ $2^{x \log x}$\\
$3^x$ \\
$x^2 + 2^x \ \ \ 2^x$ \\
$x^2 + x^3 \ \ \ x(1 + 2 + \ldots + x)$ \\
$x \log_2 x \ \ \ \log_2 x!$ \\
$\frac{x}{\log_2 x}$ \\
$\log^2 x$ \\
$\log x^2 \ \ \ \log_2 x \ \ \ \ln x$ \\
$\log \log x $ \\
\end{center}

\medskip

{\bf \item Compute the sum of the infinite series below:}
\begin{enumerate}
\item {\bf $1 + \frac{1}{4} + \frac{1}{16} + \frac{1}{64} + \ldots$}

Using the first formula in Table 2 on page 76 of the text (Rosen), we have a ratio series with $a=1, r=\frac{1}{4}$:

\begin{eqnarray*}
1 + \frac{1}{4} + \frac{1}{16} + \frac{1}{64} + \ldots & = & \sum_{k=0}^\infty (\frac{1}{4})^k \\
& = & \frac{(\frac{1}{4})^\infty - 1}{\frac{1}{4} - 1} \\
& = & \frac{-1}{-\frac{3}{4}} \\
& = & \frac{4}{3}
\end{eqnarray*}

\item {\bf $1 + \frac{2}{4} + \frac{3}{16} + \frac{4}{64} + \ldots$}

Letting $S = 1 + \frac{1}{4} + \frac{1}{16} + \frac{1}{64} + \ldots =
\frac{4}{3}$ (the sum we computes in the first part of this problem), we can rewrite
the current series as follows:

$
\begin{array}{lcrrrrl}
1 + \frac{2}{4} + \frac{3}{16} + \frac{4}{64} + \ldots & = & (1 & +\frac{1}{4} & + \frac{1}{16} & + \frac{1}{64} & + \ldots) \\
& & & +(\frac{1}{4} & + \frac{1}{16} & + \frac{1}{64} & + \ldots) \\
& & & & + (\frac{1}{16} & + \frac{1}{64} & + \ldots) \\
& & & & & + (\frac{1}{64} & + \ldots) \\
& & & & & & + \ldots \\
& = & (1 & + \frac{1}{4} & + \frac{1}{16} & + \frac{1}{64} & + \ldots) \\
&  & +\frac{1}{4}(1 & + \frac{1}{4} & + \frac{1}{16} & + \frac{1}{64} & + \ldots) \\
&  & +\frac{1}{16}(1 & + \frac{1}{4} & + \frac{1}{16} & + \frac{1}{64} & + \ldots) \\
&  & +\frac{1}{64}(1 & + \frac{1}{4} & + \frac{1}{16} & + \frac{1}{64} & + \ldots) \\
& & & & & & + \ldots \\
& = & \multicolumn{5}{l}{S + \frac{1}{4}S + \frac{1}{16}S + \frac{1}{64}S + \ldots} \\
& = & \multicolumn{5}{l}{S(1 + \frac{1}{4} + \frac{1}{16} + \frac{1}{64} + \ldots)} \\
& = & \multicolumn{5}{l}{S^2} \\
& = & \multicolumn{5}{l}{\frac{16}{9}} \\
\end{array}
$
\end{enumerate}

\medskip

\item {\bf Telescoping Series.}
\begin{enumerate}
\item {\bf Using the identity
$$
\frac{1}{k(k+1)} = \frac{1}{k} - \frac{1}{k+1}
$$
find the value of the infinite sum $\frac{1}{1*2} + \frac{1}{2*3} + \frac{1}{3*4} + \ldots$}

\begin{eqnarray*}
\frac{1}{1*2} + \frac{1}{2*3} + \frac{1}{3*4} + \ldots & = & (1 - \frac{1}{2}) + (\frac{1}{2} - \frac{1}{3}) + (\frac{1}{3} - \frac{1}{4}) + \ldots \\
& = & 1 - \frac{1}{2} + \frac{1}{2} - \frac{1}{3} + \frac{1}{3} - \frac{1}{4} + \ldots \\
& = & 1 \\
\end{eqnarray*}

\medskip

\item {\bf The $n$th triangle number is defined to be the sum of the first $n$ positive integers.  What is the value of the infinite sum of the reciprocals of the triangle numbers?}

The $n$th triangle number is $\frac{n(n+1)}{2}$.  Therefore, the
infinite sum of the reciprocals of the triangle numbers is simply
double the series above, or 2.

\end{enumerate}

\medskip

\item {\bf Countability Proofs.}
\begin{enumerate}
\item {\bf Prove that the positive odd numbers have the same cardinality as the positive even numbers.}
We show that the positive odd numbers have the same cardinality as the positive even numbers simply by nothing that the function $f(x) = x+1$ is a one-to-one mapping from the positive odd numbers to the positive even numbers.

\item {\bf Prove that the set of all integer points in the positive quadrant of 3-dimensional space are countable.}
This will follow immediately from the next part of the problem (below).


\item {\bf Prove by induction that the set of all vectors in $k$ dimensions with positive integer values is countable.}

As our base case, we use $k = 1$: this is the natural numbers, which
are countable.  Now assume that the set $V_k$ of all $k$ dimensional
vectors with positive integer values is countable.  Then there exists
some enumeration of this $V_k$; denote the $i$th element in this
enumeration by $V_{k_i}$.  We now show how to construct an enumeration
$V_{k+1}$.  We consider a two dimensional table (infinite in both
directions), with the enumeration $V_k$ along one axis and the natural
numbers along the other axis.  We then proceed ``triangularly''
through this table, taking the upper-left most entry, then the two
entries along the ``diagonal'' next to that entry, then the three
entries along the next diagonal, and so forth.  The first few elements
of our enumeration fo $V_{k+1}$ are $(V_{k_1}, 1), (V_{k_2}, 1),
(V_{k_1}, 2), (V_{k_3}, 1), (V_{k_2}, 2), (V_{k_3}, 1), \ldots$


\item {\bf Prove that the number of scheme programs is countable.  Hint: Order them by the number of characters each contains.}

We order the scheme programs by the number of characters each
contains.  Because scheme programs are composed from a finite
alphabet, for any fixed length, there are a finite number of scheme
programs of that length.  The set of all scheme programs is the union
of the set of scheme programs of length $k$, for all positive integers
$k$.  So the set of all scheme programs is the union of a countable
number of finite sets, which is countable.
\end{enumerate}

\medskip

\item {\bf The following is a version of Russell's Paradox for sets, described in your text (Rosen) in problem 26, on page 45.  Consider any computer program that takes other programs as inputs, and outputs {\em yes} or {\em no} based on some criterion.  (A compiler or interpreter, for example).  It is possible for such a program to be fed back into itself, and depending on the program it might either say {\em yes} (a Scheme interpreter written in Scheme) or {\em no} (a Scheme interpreter written in Java).  Call the programs that say {\em no} to themselves {\em self-hating} programs.  Now assume that there is a computer program that takes other programs as inputs and says {\em yes} to all the {\em self-hating} programs, and no otherwise.}

\begin{enumerate}
\item {\bf Assuming this program never runs forever on an input, analyze what happens when this program is input to itself.}

Consider first the possibility that our hypothetical program $P$ says
{\em yes} when given itself as input.  Then, since our program is
defined to be a program that says {\em yes} only to self-hating
programs, we can conclude that $P$ is self-hating.  But the definition
of self-hating programs is that they say {\em no} when given
themselves as input.  So we conclude that $P$ cannot say yes to
itself.

Now consider the possibility that $P$ says {\em no} to itself.  Then,
by how $P$ was defined, we can conclude that $P$ is not self-hating.
But if $P$ is not self-hating, then it must say {\em yes} when given
itself as input.  So $P$ cannot say {\em no} to itself either.

We conclude that there cannot exist a program which halts on all
inputs, and says yes to all self-hating programs and no to all other
programs.

\item {\bf Consider the possibility that the program will run forever answering neither {\em yes} nor {\em no} on some input.}

This offers a way out of the dilemma posed above.  The program $P$
could run forever if given itself as input.  It is possible to make a
program that with the property that if it says {\em yes} to an input
$I$, then $I$ represents a self-hating program, and if it says {\em
no} to an input $I$, then $I$ does not represent a self-hating
program, {\em if we allow that the machine may run forever on some
instances.}  A trivial example is a program that runs forever on {\em
all} instances --- it satisfies the requirements, although it is
obviously not very useful.
\end{enumerate}

\medskip

\item {\bf Counting each arithmetic calculation or comparison, extraction or exchange of a card as one operation, what is the worse-case order of growth of an algorithm that sorts numbered cards in the following way?}

\begin{enumerate}
\item {\bf Find the largest valued card in the deck by shuffling through one card at a time extracting a card if it is the largest one seen so far, and swapping the previously largest card back into the deck.  When the largest has been found, place this card face down in a new pile and repeat the previous process until no cards in the original pile are left.  Explain your answer.}

Each time we make a pass through the deck, we extract the largest
card.  Each pass through the deck takes $O(k)$ time, where $k$ is the
number of cards remaining in the deck during that pass.  The total
number of times for all the passes is:

\begin{eqnarray*}
\Theta(n + (n-1) + (n-2) + (n-3) + \ldots + 2 + 1) & = & \Theta(\frac{n(n+1)}{2}) \\
& = & \Theta(n^2)
\end{eqnarray*}

\item
{\bf This time we assume that the largest number on any of the $n$
cards is $n^2$.  We sort the cards by placing a set of $n^2$ cards
numbered from 1 to $n^2$ on a table.  Then one by one, place each
card on top of the number equal to it on the desk.  The sorted
list can be extracted by looking through all $n^2$ piles in order.
This method can be improved to work in linear time.  Explain how.
Hint: This is not easy.  Use division, to try to turn each number
into a pair of numbers, each with a value between $1$ and $n$.}

We begin by creating slots numbered 1 to $n$ on the table.  We next
divide each of our numbers by $n$, saving both the quotient and the
remainder.  Then, we place each number in the slot that corresponds to
its {\em remainder}.  We pick up the cards, starting from slot 1 and
proceeding to slot $n$.  At this point, we have the cards sorted in
order of increasing remainder --- cards that have the same remainder
are in no particular order.  We call this step the {\em remainder
pass}.

Next, we place the cards in the slots again, this time placing each
card in the slot corresponding to its {\em quotient}.  We proceed
through the slots again, picking up the cards, {\em putting cards that
were placed in the slot first before cards that were placed in the
slot later}.  We call this step the {\em quotient pass}.  After the
quotient pass, we are done.

It should be clear that this algorithm operates in linear time --- we
perform a constant number of steps, each of which take linear time.
Does it work?  Are the cards sorted after the second pass?  Yes.
Clearly, the second pass will order any two cards with different
quotients properly (in fact, if we didn't care about getting a
complete ordering, but only cared about getting cards with different
quotients in the correct order, we could just do the second pass,
ignoring the initial, remainder pass).  Now, consider any two cards
with the same quotient, but different remainders.  The card with the
smaller remainder will be earlier in the pile after the remainder
pass, so it will be placed in the appropriate quotient pile {\em
before} the card with the larger remainder.

\end{enumerate}

\end{enumerate}
\end{document}
