%&LaTeX
\documentclass{article}
\usepackage{hyperref}
\def\R2Lurl#1#2{\mbox{\href{#1}{\tt #2}}}                                    
                                                                                                    
                                                                                                    
                                                                                                    
                                                                                                    
                                                                                                    
                                                                                                    
                                                                                                    
                                                                                                    
                                                                                                    
\newcommand{\tab}{\makebox[4em]{}}


\begin{document}


\begin{center}
\textbf{ArsDigita University}


\end{center}

\begin{center}
\textbf{Month 2: Discrete Mathematics - Professor Shai Simonson}


\end{center}


\begin{center}
\textbf{Problem Set 3 -- Recursion and Induction}


\end{center}

1.\tab 
Solve the Chinese Rings Puzzle, also called the Patience Puzzle, 
(\R2Lurl{http://johnrausch.com/PuzzleWorld/patience.htm }{http://johnrausch.com/PuzzleWorld/patience.htm}). 
This will prepare you for recursion, recurrence equations, proofs 
by induction and graph representations.


2.\tab 
Consider the variation of the Towers of Hanoi Problem where you 
have four pegs instead of three. Sloppy Joe designs this \textit{solution:}



In order to move \textit{n} disks from \textit{From} to \textit{To}, using \textit{Using1} and 
\textit{Using2}: 


If \textit{n} equals 1, then move a disk from \textit{From} to \textit{To,} otherwise 
do the three recursive steps:


Move \textit{n/2} disks from \textit{From} to \textit{Using1}, using \textit{To} and \textit{Using2;}


Move \textit{n/2} disks from \textit{From} to \textit{To}, using \textit{Using2} and \textit{Using1;}\\
Move \textit{n/2} disks from \textit{Using1} to \textit{To}, using \textit{From} and \textit{Using2;}\\
a.\tab 
Explain why Sloppy Joe's solution does not work.\\
b.\tab Fruity Freddie suggests changing the second line: 


Move \textit{n/2} disks from \textit{From} to \textit{To}, using \textit{Using2} and \textit{Using1;} to\\
Move \textit{n/2} disks from \textit{From} to \textit{To}, using \textit{Using1} and \textit{Using2;}


Explain why the algorithm still does not work.\\
c.\tab 
Code the algorithms in Scheme and report what happens for \textit{n} = 
3, and \textit{n =} 4.\\
d.\tab If either one of the solutions above really worked, then set 
up a recurrence equation for the number of moves it takes to 
solve the problem.\\
e.\tab Solve this recurrence equation by repeated substitution.


3.\tab 
Fix Joe and Freddie's failures.\\
a.\tab 
Construct a correct solution to the Towers of Hanoi problem with 
four pegs. \\
b.\tab Code your solution in Scheme and show the result for \textit{n 
=} 3 and \textit{n =} 4\textit{.}\\
c.\tab Construct a recurrence equation for your solution and solve 
it.


4.\tab 
Make a picture of the Towers of Hanoi graph for \textit{n =} 4. Recall 
that this graph is composed of three copies of the Hanoi graph 
for 3 disks.


5.\tab 
In the original Towers of Hanoi problem, alternately color the 
disks black and white as you go from the bottom to the top of 
the starting tower. Then in addition to the standard rules, add 
the constraint that only disks of different colors can be placed 
on top of one another.


a.\tab 
Show the path of the solution in the Towers of Hanoi graph for \textit{n 
=} 4. Show that aside from going backwards, there is no choice 
for each subsequent move.\\
b.\tab Prove by induction on the number of disks \textit{n}, that the solution 
implied by this new rule, is identical to the standard recursive 
solution. Hint: First prove the following lemma by induction: 
In the Hanoi graph on \textit{n} disks, the rightmost occurrence of \textit{a} and \textit{b,} in 
any node on the solution path from node \textit{a}$^{\mathit{n}}$ to \textit{b}$^{\mathit{n}}$\textit{,} occur 
in odd numbered slots counting from right to left. \\
6.\tab 
In the original Towers of Hanoi problem, add the constraint that 
no direct moves from the \textit{From} peg to the \textit{To} peg are allowed.


a.\tab 
Prove by induction, that following this new rule, will take you 
through every legal configuration of the game. Hint: Use the 
graph representation.\\
b.\tab Write a recurrence equation for the smallest number of moves 
it takes to solve the problem under the new constraint.\\
c.\tab Solve the equation.


7.\tab 
Double-disk Hanoi has three pegs and \textit{n} disks but there are 
two identical copies of each disk stacked on top of each other. 
The goal is the same as before, but we allow disks of the same 
size to reverse their order.


a.\tab 
Describe a method to solve this problem.\\
b.\tab Write and solve a recurrence equation for the number of moves 
implied by your solution.


8.\tab 
Consider the standard Gray code sequence for binary numbers of 
length 4.


a.\tab 
Write down the standard Gray codes in order for binary numbers 
of length four.\\
b.\tab Draw a four-dimensional hypercube, labeling the vertices appropriately, 
and show that the Gray code ordering is a Hamiltonian Circuit 
through the Hypercube.


9.\tab 
Prove by induction that the \textit{reverse} method for generating \textit{n-}bit 
Gray codes from \textit{n-1} bit Gray codes is a Hamiltonian Circuit 
through the \textit{n}-dimensional hypercube. 


10.\tab 
 Make a table of the \textit{2}$^{\mathit{n}}$ Gray codes for some \textit{n,} by 
writing them in order starting from \textit{0}$^{\mathit{n}}$, one underneath 
the other. 


a.\tab 
Conjecture a theorem regarding the patterns of the bits in the \textit{i}th 
column of your chart. \\
b.\tab 
Prove your theorem.


11.\tab 
Converting Gray codes.


a.\tab 
Write a Scheme program that takes a binary number \textit{n} and outputs 
the \textit{n}th Gray code.\\
b.\tab Write a Scheme program that does the inverse of (a).


12.\tab 
Consider the algorithm for computing \textit{a}$^{\mathit{n}}$ where \textit{a} and \textit{n} are 
integers. If \textit{n} = 0, then return 1. If \textit{n} is even then compute \textit{a}$^{\mathit{n/2}}$ recursively, 
and square it. Otherwise, compute \textit{a}$^{\mathit{n-1}}$ recursively and 
multiply the result by \textit{a.}


a.\tab 
How many multiplications does this method use when \textit{n} is a 
power of two?\\
b.\tab How many multiplications when \textit{n} is one less than a power 
of two?\\
c.\tab What exactly determines the number of multiplications for general \textit{n.} Be 
as specific as possible.




\end{document}
