\documentclass[12pt]{article}
\title{Card Trick Problem Set Solutions}
\usepackage{fullpage}
\usepackage[dvips]{graphicx}
\author{Michael Allen}

\begin{document}
\maketitle

\section*{1-4 Lower Bounds}

With the given procedure, we can establish a lower bound for the
size of the deck for which we can perform the card trick for a
given number of cards. These results are shown below.

$$\begin{array}{c|c}
    hand&   deck \\\hline
    2   &   3 \\
    3   &   6 \\
    4   &   15 \\
    5   &   52 \\
    6   &   245 \\
    n   &   2(n-1)! + (n-1)
  \end{array}$$

\section*{5-6 Upper Bounds}

Our procedure, however, is not optimal, and it is theoretically
possible to perform the trick with a larger deck. We can set our
upper bound at a size which makes maximal use of the information
available to us.

$$\begin{array}{c|c}
    hand&   deck \\\hline
    2   &   3 \\
    3   &   8 \\
    4   &   24 \\
    5   &   124 \\
    6   &   725 \\
    n   &   n! + (n-1)
  \end{array}$$

\newpage
\section*{7}

For the following, assume we have a deck of $d$ cards from which
the accomplice chooses a hand of size $n$.

{\bf a.} The number of ways the accomplice can choose a hand is
$C(d,n)$.

{\bf b.} The number of different permutations of the n-1 cards the
accomplice can show is $n!$.

{\bf c.} The number of different strategies available to the
accomplice and magician are $n!^{C(d,n)}$

\section*{8a}

We can verify that a strategy is successful by showing that each
possible hand maps to a different ordering chosen by the
accomplice. The table below shows how this can be done for a deck
with two suits ($a$ and $b$) of three cards each where three cards
are chosen .

$$\begin{array} {ccc c ccc | ccc c ccc}
    &hand& &\rightarrow& 1&2&hide & &hand& &\rightarrow& 1&2&hide\\ \hline
    1a&2a&3a &\rightarrow& 1a&3a&2a & 2a&3a&1b &\rightarrow& 2a&1b&3a \\
    1a&2a&1b &\rightarrow& 1a&1b&2a & 2a&3a&2b &\rightarrow& 2a&2b&3a \\
    1a&2a&2b &\rightarrow& 1a&2b&2a & 2a&3a&3b &\rightarrow& 2a&3b&3a \\
    1a&2a&3b &\rightarrow& 1a&3b&2a & 2a&1b&2b &\rightarrow& 1b&2a&2b \\
    1a&3a&1b &\rightarrow& 3a&1b&1a & 2a&1b&3b &\rightarrow& 3b&2a&1b \\
    1a&3a&2b &\rightarrow& 3a&2b&1a & 2a&2b&3b &\rightarrow& 2b&2a&3b \\
    1a&3a&3b &\rightarrow& 3a&3b&1a & 3a&1b&2b &\rightarrow& 1b&3a&2b \\
    1a&1b&2b &\rightarrow& 1b&1a&2b & 3a&1b&3b &\rightarrow& 3b&3a&1b \\
    1a&1b&3b &\rightarrow& 3b&1a&1b & 3a&2b&3b &\rightarrow& 2b&3a&3b \\
    1a&2b&3b &\rightarrow& 2b&1a&3b & 1b&2b&3b &\rightarrow& 1b&3b&2b
  \end{array}$$

\section*{8b}

If we try to add another card to our deck, say $4a$, we find that
there is no way to encode the hands, $(1a,3a,$[any b]$)$ and
$(2a,4a,$[any b]$)$ with the rules we are given. Specifically, we
cannot hide the higher of two consecutive cards (where the lowest
is one greater than the highest) of the same suit because the two
$a$ cards are not consecutive.

\newpage
\section*{9}

We can represent strategies for the card trick in the form of an
$(n-1)$-dimensional table which shows the magician how to decode
the hidden card. Then, developing a strategy is just a matter of
fitting one of the permutations of every possible hand into the
table. The $n=3,d=7$ case can be computed by hand. The table for
the $n=3,d=8$ case is shown below.

$$\begin{array} {c|cccccccc}
     & 0&1&2&3&4&5&6&7 \\
     \hline
    0&  &3&4&5&5&2&2&2 \\
    1& 2& &4&5&6&6&3&3 \\
    2& 3&3& &5&6&7&7&4 \\
    3& 4&4&4& &6&7&7&4 \\
    4& 1&5&5&5& &7&7&1 \\
    5& 1&2&6&6&6& &7&1 \\
    6& 1&2&3&0&0&0& &1 \\
    7& 1&2&3&0&0&0&0&
  \end{array}$$

\section*{10}

Generating strategies for each size of hand and deck can be
accomplished by a brute force search with a computer, but there is
also a clever closed form solution. It is possible (though
difficult) to prove that for any hand size $n$, the upper bounds
as specified in {\em problems 5-6} can be met.

\end{document}
