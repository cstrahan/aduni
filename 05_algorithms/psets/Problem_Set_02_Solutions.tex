\documentclass{article}

\usepackage{fullpage}
\usepackage{epsfig}

\title{Algorithms - Problem Set 2 Solutions}
\author{by Michael Allen}

\begin{document}

\maketitle

\begin{enumerate}
\item {\bf Practice with Red-Black Trees}

\begin{enumerate}
\item The successive Red-Black trees that are constructed by
adding the keys 10, 20, 30, 40, 50, 60, 70, 80, 90, 100 into an
initially empty tree are shown in figure \ref{2-1a}.

\begin{figure}[htbp]
  \centerline{\psfig{file=2-1a.eps}}
  \caption{building a red-black tree}
  \label{2-1a}
\end{figure}

\item The results of deleting the values 60, 70, and 90 are shown
in figure \ref{2-1b}. On the left are the results if we do not
maintain the red-black property, and the right shows if we do.

\begin{figure}[htbp]
  \centerline{\psfig{file=2-1b.eps}}
  \caption{after elements are deleted}
  \label{2-1b}
\end{figure}
\end{enumerate}

\item {\bf Practice with Graph Algorithms}
\begin{enumerate}
\item The graph shown in figure \ref{2-2a} will yield incorrect
results using Dijkstra's algorithm.

\begin{figure}[htbp]
  \centerline{\psfig{file=2-2a.eps}}
  \caption{Dijkstra can't handle this graph}
  \label{2-2a}
\end{figure}

\item The depth-first search tree  with back edges for the interval graph
containing \{[1,3], [2,4], [6,9], [5,10], [8,9], [8,11], [3,6],
[1,4], [3,7] \} is shown in figure \ref{2-2b}.

\begin{figure}[htbp]
  \centerline{\psfig{file=2-2b.eps}}
  \caption{interval graph DFS tree}
  \label{2-2b}
\end{figure}

\item The shortest path tree using Dijkstra's algorithm, and the
breadth first search tree for the graph on page 499 of the text
are shown in figure \ref{2-2c}.

\begin{figure}[htbp]
  \centerline{\psfig{file=2-2c.eps}}
  \caption{shortest path and search trees}
  \label{2-2c}
\end{figure}
\end{enumerate}

\item {\bf Shortest Path Algorithm}

The BFS-scanning shortest path algorithm will go into an infinite
loop on a graph with a negative cost cycle. Nodes that are a part
of the cycle will be repeatedly explored as the distances are
relaxed more and more.

\item{\bf Topological Sorting}

{\it no code yet}

\item {\bf Finding a Cycle in an Undirected Graph}

The algorithm below can be used to find and print a cycle in an
undirected graph. Works very much like DFS, except that when it
encounters a node it has already searched (which only happens when
a cycle is present), it prints out the cycle.

\begin{verbatim}
FindCycle(G)
1. for each vertex u in V[G]
2.   do color[u] <- white
3.      parent[u] <- nil
3. depth <- 0
4. for each vertex u in V[G]
5.   do if color[u] = white
6.        then FC-Visit(u)

FC-Visit(u)
1. color[u] = gray
2. for each v in Adj[u]
3.   do if color[v] = gray and v != parent[u]
4.        then parent[v] = u
5.             FC-PrintOut(v)
6.      if color[v] = white
7.        then parent[v] = u
8.             FC-Visit(v)
9. color[u] = black

FC-PrintOut(v)
1. w <- v
2. do w <- parent[w]
3.    print w
3.   while w != v
4. end
\end{verbatim}

\item {\bf Depth First Search}

{\it no code yet}

\item {\bf The Circus Problem}

Given the heights and weights of a number of performers, the
following algorithm will find the tallest stack that can be built
with no performer standing on the shoulders of another who is
lighter or shorter.

\begin{enumerate}
\item Add a node to the graph for each performer.
\item Examine each pair of nodes and add a directed edge from A to
B if A can stand of B's shoulders.
\item Add a "start" node with a directed edge to every other node
in the graph.
\item Find the longest path starting at the start node. (This can
be accomplished by assign each edge a weight of -1 and finding the
shortest path, since we are guaranteed not to get a negative
cycle.)
\item Find the farthest node and rebuild the path by following the
appropriate edges backwards.
\end{enumerate}

\end{enumerate}
\end{document}
