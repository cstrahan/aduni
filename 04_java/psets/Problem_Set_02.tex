\documentclass{article}
\setlength{\textwidth}{6.0in}
\setlength{\textheight}{9.0in}
\setlength{\headheight}{0.0in}
\setlength{\headsep}{0.0in}
\setlength{\parskip}{1.5ex plus 0.5ex minus 0.5ex}
\setlength{\topmargin}{0.5in}
\setlength{\topskip}{0.0in}
\setlength{\evensidemargin}{0.5in}
\setlength{\oddsidemargin}{0.5in}
\newcounter{exercise}
\newcounter{problem}
\newcounter{step}[problem]
\newcounter{subproblem}[problem]
\newcommand {\exercise}[1]{\bigskip\noindent{\large\bf Exercise \stepcounter{exercise} \arabic{exercise}: {#1}}}
\newcommand {\problem}[1]{\bigskip\noindent{\large\bf Problem \stepcounter{problem} \arabic{problem}: {#1}}}
\newcommand {\step}{\par\noindent{\bf Step \stepcounter{step} \arabic{step}: }}

\usepackage[dvips]{graphics,color}


\begin{document}


\begin{center}
\LARGE
Object-Oriented Programming in Java
\end{center}
\bigskip
{\noindent\Large\bf Problem Set 2 \hfill Due: Jan 17, 2001}
\hrule
\bigskip
\noindent{\Large\bf Exercises}
\bigskip

\exercise{Goodbye World}

Write a program that opens a window, prints ``Goodbye World'' in
it, and exits properly when the close window ('x') icon on the
window title bar is clicked. 

First try doing it using the 
{\tt setDefaultCloseOperation(JFrame.EXIT\_ON\_CLOSE);} method as described
in Chapter 7. If you are ambitious, try catching the WindowEvents 
following the code developed in Chapter 8 (page 364+). 
Try using anonymous inner classes in your implementation.

\exercise{Threads}

Extend the {\tt Thread} class to write two classes, {\tt Walk} and
{\tt Chew}. Have {\tt run} method of {\tt Walk} print out
``left'' then ``right'' forever. Have {\tt run} method of {\tt Chew} print out
``chomp'' forever. Allocate and start both {\tt Walk} and {\tt Chew} and
examine the output. You will have to use control-c to quit.

[To give credit where it is due: This exercise was strongly inspired by
{\it Mr. Bunny's Bug Cup O' Java}]

\bigskip
\noindent{\Large\bf Problems}
\bigskip

\problem{File I/O and Exception Handling}

The Caesar Cipher is a (very insecure) method for encrypting text
dating back to the Romans. It is a simple alphabetic shift cypher
that replaces each letter in the text to be encoded with the
letter 3 later in the alphabet. For example, 'A' gets coded as 'D'
and 'L' gets coded to 'O'. Letters at the end of the alphabet wrap
around to the beginning, so 'Z' gets coded as 'C'. To decode the message, one
simply reverses the shift. 

Java will let you treat 'char' variables as number under some circumstances,
you can shift by {\tt n} characters by adding {\tt n} to a character.
For example:

\begin{verbatim}

int shift = 3; // for example
char input - 'c';
char output = input + 3; // output <- 'f'

\end{verbatim}


One can generalize this technique by
using a shift other than 3, the value of the shift then becomes the
cipher 'key'. One can also generalize to handle characters other than
capital letters. Most of the interesting printable characters lie
between ' ' (ASCII 32) and '~' (ASCII 126). If we shift within this
space, we can encode most messages. The shift algorithm now becomes:
\begin{verbatim}
int temp = input + shift;  // compute new char
// handle wrap around (127 == '~', 32 == ' ')
if(temp >= 127) temp = 32 + (temp - 127);
char output = (char) temp; // convert back to 'char'
\end{verbatim}

Write a Java program Encode with command line usage:

\begin{verbatim}
java Encode key infile [outfile]
\end{verbatim}

This program opens the file $infile$, reads and encrypts each line
using the shift $key$ (a number), then writes each line to $outfile$.
If $outfile$ is not given the program should print to the console.

The program should be robust to error conditions including:
\begin{itemize}
\setlength{\parskip}{0pt}
\item Wrong number or type of arguments given (for example, no infile or
a key that is not a number or too big),
\item Infile not found or readable,
\item Outfile not found or writable,
\item Infile is binary or contains non-printable characters.
\end{itemize}

By robust, we mean the program should exit gracefully and print out a
useful error message.

Note: This problem is about opening, reading, writing, and closing text files,
and handling error conditions and exceptions gracefully. If you find the
encrypting a distraction, skip it and just copy the files verbatim.
If, on the other hand, you enjoy it, try extending your program to take a 
word as a key, using each letter
of the keyword in turn as the start of the shift. For example, if the keyword
is ``cat'' the first letter of the message is encoded with shift ('c' - ' '),
the second with shift ('a' - ' '), and so on. When the end of keyword is
reached, start over at the beginning. (This is a variant on the
Vigenere Cipher, somewhat more secure than the simple shift, but nowhere
near good enough to befuddle the NSA.)

\problem{SameJava}

Below is a screen-shot of the Gnome game ``Same Gnome'', available
as part of the gnome-games package. The goal of this problem is
to implement this game in Java, though with less ambitious rendering
of the balls.

\begin{center}
\begin{figure}[h]
\begin{center}
\resizebox{4.0in}{!}{
\includegraphics{sg.ps}
}
\caption[Figure 1]{Screenshot of Same Gnome}
\end{center}
\end{figure}
\end{center}

The game board consists of a 15 by 10 array of squares. Each square can
contain a ball of one of three colors. The board is initialized by placing
a ball in each square, choosing its color at random. When the player moves
the mouse cursor over a ball, that ball and all connected balls of
the same color are highlighted. ( Two balls are connected if they share a
side, or share a side with a another ball connected to either. In other
words, if two balls are connected, there is a path from one to the other
that remains on the same color). Isolated balls of any color are not
activated by the mouse nor deleted.

If the player clicks the mouse, the connected set of balls in removed from
the board and the board is compacted. The board is compacted by first
dropping balls within columns to fill in empty spaces (as if under the
influence of gravity). The board is then compacted by sliding complete
columns to the left to remove and empty columns. Each time the player
clicks the mouse, their score it incremented by the {\it square} of the
number of balls deleted by that click.

The game ends when there are no more groups of balls to click on.

\bigskip
\noindent{\bf Design and Implementation strategy}

Since this is a substantial project, one should approach the design
systematically rather than plunging into implementation. We recommend
following closely the steps below. 

In addition, we will try to supply,
for those who want to use it, a framework that has mostly the
standard code necessary start up this type of graphics application,
as well as some (ugly) display methods to get you started.

\step
Write out, in outline form, in English, what is happening during the play
up a game.

\step
Ignoring the display operations, decide on what classes are required
to implement the game. 

\step
Decide on what methods these classes need to implement. Write out
the method declarations (the name of the method, its arguments, and its
return type). Leave the implementations empty, but include JavaDoc
comments.

\step
Decide on what data belongs on each object and how it will be represented.
Add appropriate instance variables.

\step
Talk to a TA about your design.

\step
Implement the methods in a reasonable order to allow for
thorough testing. A good strategy is:
\begin{itemize}
\setlength{\parskip}{0pt}
\item Implement constructors and methods to initialize the board.
\item Main() and window initialization (use our version if you like).
\item Methods to display the board. Start with simple methods that
draw the balls as circles and highlight them by turning them white.
\item Code to handle mouse event and highlight a ball when
mouse passes over.
\item Methods to compute and highlight the connected set of selected
balls. [Note: This is probably the hairiest method to implement. My advice
is to write a version that selects only the Ball that the mouse
is over. The rest of the game can be then be finished. Once everything
else is working, improve this method to find the connected subset.]
\item Methods to delete balls and compress board.
\end{itemize}

\step
Once everything is working (and if you have the time and inclination),
there are several ways to spiff up the game.

\begin{itemize}
\setlength{\parskip}{0pt}
\item Better rendering of balls. Fancier graphics or images.
\item Better highlighting. For example, some animation ( a good chance
to experiment with threads or timers).
\item Display current score and keep top ten scores.
\item Whatever else you think is cool..
\end{itemize}

\end{document}




